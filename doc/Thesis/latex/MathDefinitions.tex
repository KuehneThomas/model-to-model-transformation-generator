%Mathe
\usepackage{amsmath}
\usepackage{amsfonts}

\usepackage{bbm}
\newcommand{\IN}{\mathbbm{N}}           % nat�rliche Zahlen
\newcommand{\IZ}{\mathbbm{Z}}           % ganze      Zahlen
\newcommand{\IQ}{\mathbbm{Q}}           % rationale  Zahlen
\newcommand{\IR}{\mathbbm{R}}           % reelle     Zahlen
\newcommand{\IC}{\mathbbm{C}}           % komplexe   Zahlen
\newcommand{\IP}{\mathbbm{P}}           % Primzahlen

\renewcommand{\epsilon}{\varepsilon}                    % sch�neres Epsilon

\newcommand{\set}[1]{\left\{ #1 \right\}}               % Menge
\newcommand{\powerset}[1]{\wp\!\left(#1\right)}         % Potenzmenge
\newcommand{\abs}[1]{\left\lvert #1 \right\rvert}       % Betrag
\newcommand{\norm}[1]{\left\lVert #1 \right\rVert}      % Norm
\newcommand{\floor}[1]{\left\lfloor #1 \right\rfloor}   % floor
\newcommand{\ceil}[1]{\left\lceil #1 \right\rceil}      % ceiling

\newcommand{\signatur}[3]{#1:#2\,\rightarrow\,#3}     %Funktionen-Signatur


%Umgebungen
\newenvironment{bew}{\vspace{6pt}{\raggedright\textbf{Beweis: }}} {\newline\vspace{6pt}\hfill{$\square$}\vspace{6pt}\newline}

\usepackage{theorem}
%\theoremstyle{change} %vertauscht Nummer und Titel, z.B. statt "Def. 1.1" steht "1.1 Def."
\newtheorem{definition}{Definition}[chapter]
\newtheorem{satz}[definition]{Satz}
\newtheorem{korollar}[definition]{Korollar}
\newtheorem{lemma}[definition]{Lemma}
\newtheorem{theorem}[definition]{Theorem}
\newtheorem{bsp}[definition]{Beispiel}
{\theoremstyle{changebreak}\newtheorem{bspe}[definition]{Beispiele}} %mit neuer Zeile anfangen
